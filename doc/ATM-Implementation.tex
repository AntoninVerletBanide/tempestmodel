\documentclass{article}

\makeatletter
\renewcommand\section{\@startsection
  {section}{2}{0mm}%name, level, indent
  {-\baselineskip}%             beforeskip
  {0.5\baselineskip}%            afterskip
  {\normalfont\Huge\bfseries}}% style
\makeatother

\usepackage{amsmath}
\usepackage{graphicx}
\usepackage{multicol}

\oddsidemargin 0cm
\evensidemargin 0cm

\textwidth 16.5cm
\topmargin -2.0cm
\parindent 0cm
\textheight 24cm
\parskip 0.5cm

\usepackage{fancyhdr}
\pagestyle{fancy}
\fancyhf{}
%\fancyhead[L]{AOSS Reference Sheet}
%\fancyhead[CH]{test}
\fancyfoot[C]{Page \thepage}
                                 
\newcommand{\vb}{\mathbf}
\newcommand{\vg}{\boldsymbol}
\newcommand{\mat}{\mathsf}
\newcommand{\diff}[2]{\frac{d #1}{d #2}}
\newcommand{\diffsq}[2]{\frac{d^2 #1}{{d #2}^2}}
\newcommand{\pdiff}[2]{\frac{\partial #1}{\partial #2}}
\newcommand{\pdiffsq}[2]{\frac{\partial^2 #1}{{\partial #2}^2}}
\newcommand{\topic}{\textbf}
\newcommand{\arccot}{\mathrm{arccot}}
\newcommand{\arcsinh}{\mathrm{arcsinh}}
\newcommand{\arccosh}{\mathrm{arccosh}}
\newcommand{\arctanh}{\mathrm{arctanh}}

\title{\Huge \textbf{HARDCore: The High-Order Adaptively Refined Dynamical Core}}
\author{\Large Paul Ullrich}
\date{September 9th, 2013}

\begin{document}

\section{Geometry}

\subsection{Terrain-Following Cartesian Geometry}

For a given vertical coordinate transform $Z = Z(\alpha, \beta, \xi)$.  For example, Gal-Chen and Somerville (1975) coordinates:
\begin{equation}
Z(\alpha, \beta, \xi) = \xi \left[ z_{top} - z_s(\alpha, \beta) \right] + z_s(\alpha, \beta).
\end{equation}

\begin{equation}
g_{ij} = \left( \begin{array}{ccc} 1 & 0 & 0 \\[2.0ex] 0 & 1 & 0 \\[2.0ex] 0 & 0 & 0 \end{array} \right) + \left( \begin{array}{c} \pdiff{Z}{\alpha} \\[2.0ex] \pdiff{Z}{\beta} \\[2.0ex] \pdiff{Z}{\xi} \end{array} \right) \left( \begin{array}{ccc} \pdiff{Z}{\alpha} & \pdiff{Z}{\beta} & \pdiff{Z}{\xi} \end{array} \right)
\end{equation}
\begin{equation}
J = \left( \pdiff{Z}{\xi} \right)
\end{equation}
\begin{equation}
g^{ij} = \left( \begin{array}{ccc} 1 & 0 & - \left( \pdiff{Z}{\xi} \right)^{-1} \left( \pdiff{Z}{\alpha} \right) \\[2.0ex] 0 & 1 & - \left( \pdiff{Z}{\xi} \right)^{-1} \left( \pdiff{Z}{\beta} \right) \\[2.0ex] - \left( \pdiff{Z}{\xi} \right)^{-1} \left( \pdiff{Z}{\alpha} \right) & - \left( \pdiff{Z}{\xi} \right)^{-1} \left( \pdiff{Z}{\beta} \right) & \left( \pdiff{Z}{\xi} \right)^{-2} \left[ 1 + \left( \pdiff{Z}{\alpha} \right)^2 + \left( \pdiff{Z}{\beta} \right)^2 \right] \end{array} \right)
\end{equation}
\begin{align}
\Gamma^{\alpha}_{\ i j} &= \vb{0}, & \Gamma^{\beta}_{\ i j} &= \vb{0}, & \Gamma^{\xi}_{\ i j} = \left( \pdiff{Z}{\xi} \right)^{-1} \left( \begin{array}{ccc} \frac{\partial^2 Z}{\partial \alpha^2} & \frac{\partial^2 Z}{\partial \alpha \partial \beta} & \frac{\partial^2 Z}{\partial \alpha \partial \xi} \\[2.0ex] \frac{\partial^2 Z}{\partial \alpha \partial \beta} & \frac{\partial^2 Z}{\partial \beta^2} & \frac{\partial^2 Z}{\partial \xi \partial \beta} \\[2.0ex] \frac{\partial^2 Z}{\partial \xi \partial \alpha} & \frac{\partial^2 Z}{\partial \xi \partial \beta} & \frac{\partial^2 Z}{\partial \xi^2} \end{array} \right) 
\end{align}

\subsection{Terrain-Following Cubed-Sphere Geometry}

For a given vertical coordinate transform $r = R(\alpha, \beta, \xi)$.  For example, Gal-Chen and Somerville (1975) coordinates:
\begin{equation}
R(\alpha, \beta, \xi) = \xi \left[ z_{top} - z_s(\alpha, \beta) \right] + a + z_s(\alpha, \beta).
\end{equation}

\begin{equation}
g_{ij} = \frac{a^2 (1+X^2) (1+Y^2)}{\delta^4} \left( \begin{array}{ccc} 1+X^2 & - X Y & 0 \\[2.0ex] - X Y & 1+Y^2 & 0 \\[2.0ex] 0 & 0 & 0 \end{array} \right) + \left( \begin{array}{c} \pdiff{R}{\alpha} \\[2.0ex] \pdiff{R}{\beta} \\[2.0ex] \pdiff{R}{\xi} \end{array} \right) \left( \begin{array}{ccc} \pdiff{R}{\alpha} & \pdiff{R}{\beta} & \pdiff{R}{\xi} \end{array} \right)
\end{equation}

\begin{equation}
J = \frac{1}{\delta^3} \left( \pdiff{R}{\xi} \right) a^2 (1+X^2) (1+Y^2)
\end{equation}

\begin{equation}
g^{ij} = \frac{\delta^2}{a^2 (1+X^2) (1+Y^2)} \left( \begin{array}{ccc} 1+Y^2 & X Y & 0 \\[2.0ex] X Y & 1+X^2 & 0 \\[2.0ex] 0 & 0 & 0 \end{array} \right) + \tilde{g}^{ij},
\end{equation}
\begin{align}
\tilde{g}^{\alpha \alpha} &= 0 \\
\tilde{g}^{\alpha \beta} &= 0 \\
\tilde{g}^{\beta \beta} &= 0 \\
\tilde{g}^{\alpha \xi} &= - \frac{\delta^2}{a^2 (1+X^2) (1+Y^2)} \left( \pdiff{R}{\xi} \right)^{-1} \left[ (1+Y^2) \left( \pdiff{R}{\alpha} \right) + X Y \left( \pdiff{R}{\beta} \right) \right] \\
\tilde{g}^{\beta \xi} &= - \frac{\delta^2}{a^2 (1+X^2) (1+Y^2)} \left( \pdiff{R}{\xi} \right)^{-1} \left[ X Y \left( \pdiff{R}{\alpha} \right) + (1+X^2) \left( \pdiff{R}{\beta} \right) \right] \\
\tilde{g}^{\xi \xi} &= \left( \pdiff{R}{\xi} \right)^{-2} + \frac{\delta^2}{a^2 (1+X^2) (1+Y^2)} \left( \pdiff{R}{\xi} \right)^{-2} \left[ (1+Y^2) \left( \pdiff{R}{\alpha} \right)^2 + 2 X Y \left( \pdiff{R}{\alpha} \right) \left( \pdiff{R}{\beta} \right) + (1+X^2) \left( \pdiff{R}{\beta} \right)^2 \right]
\end{align}

\begin{equation}
\Gamma^{\alpha}_{\ i j} = \left( \begin{array}{ccc} \displaystyle \frac{2 X Y^2}{\delta^2} & \displaystyle - \frac{Y (1+Y^2)}{\delta^2} & 0 \\[2.0ex] \displaystyle - \frac{Y (1+Y^2)}{\delta^2} & 0 & 0 \\[2.0ex] 0 & 0 & 0 \end{array} \right) \qquad \Gamma^{\beta}_{\ i j} = \left( \begin{array}{ccc} \displaystyle 0 & \displaystyle - \frac{X (1+X^2)}{\delta^2} & 0 \\[2.0ex] \displaystyle - \frac{X (1+X^2)}{\delta^2} & \displaystyle \frac{2 X^2 Y}{\delta^2} & 0 \\[2.0ex] 0 & 0 & 0 \end{array} \right)
\end{equation}

\begin{equation}
\Gamma^{\xi}_{\ i j} = \left( \pdiff{R}{\xi} \right)^{-1} \left( \begin{array}{ccc} \displaystyle - \frac{2 X Y^2}{\delta^2} \left( \pdiff{R}{\alpha} \right) + \left( \frac{\partial^2 R}{\partial \alpha^2} \right) & \displaystyle \frac{Y (1+Y^2)}{\delta^2} \left( \pdiff{R}{\alpha} \right) + \frac{X (1+X^2)}{\delta^2} \left( \pdiff{R}{\beta} \right) + \left( \frac{\partial^2 R}{\partial \alpha \partial \beta} \right) & \displaystyle \left( \frac{\partial^2 R}{\partial \alpha \partial \xi} \right) \\[4.0ex] \cdots & \displaystyle - \frac{2 X^2 Y}{\delta^2} \left( \pdiff{R}{\beta} \right) + \left( \frac{\partial^2 R}{\partial \beta^2} \right) & \displaystyle \left( \frac{\partial^2 R}{\partial \beta \partial \xi} \right) \\[4.0ex] \cdots & \cdots & \displaystyle \left( \frac{\partial^2 R}{\partial \xi^2} \right)  \end{array} \right)
\end{equation}

\section{Hydrodynamics}

The system of equations describing the hydrodynamic system in arbitrary geometry is as follows:
\begin{align}
\pdiff{u^\alpha}{t} + u^\alpha \nabla_\alpha u^\alpha + u^\beta \nabla_\beta u^\alpha + u^\xi \nabla_\xi u^\alpha + \frac{1}{\rho} \nabla^\alpha p + f g^{\alpha j} \epsilon_{j \xi k} u^k &= 0 \\
\pdiff{u^\beta}{t} + u^\alpha \nabla_\alpha u^\beta + u^\beta \nabla_\beta u^\beta + u^\xi \nabla_\xi u^\beta + \frac{1}{\rho} \nabla^\beta p + f g^{\beta j} \epsilon_{j \xi k} u^k &= 0 \\
\pdiff{\theta}{t} + u^\alpha \pdiff{\theta}{\alpha} + u^\beta \pdiff{\theta}{\beta} &= - u^\xi \pdiff{\theta}{\xi} \\
\pdiff{u^\xi}{t} + u^\alpha \nabla_\alpha u^\xi + u^\beta \nabla_\beta u^\xi + u^\xi \nabla_\xi u^\xi &= - \theta \nabla^\xi \Pi \\
\pdiff{\rho}{t} + \frac{1}{J} \pdiff{}{\alpha} (J \rho u^\alpha) + \frac{1}{J} \pdiff{}{\beta} (J \rho u^\beta) &= - \frac{1}{J} \pdiff{}{\xi} (J \rho u^\xi)
\end{align}
Advection and geometry:
\begin{align}
u^\alpha \nabla_\alpha u^k &= u^\alpha \left[ \pdiff{u^k}{\alpha} + \Gamma^k_{\ \alpha \alpha} u^\alpha + \Gamma^k_{\ \alpha \beta} u^\beta + \Gamma^k_{\ \alpha \xi} u^\xi \right] \\
u^\beta \nabla_\beta u^k &= u^\beta \left[ \pdiff{u^k}{\beta} + \Gamma^k_{\ \beta \alpha} u^\alpha + \Gamma^k_{\ \beta \beta} u^\beta + \Gamma^k_{\ \beta \xi} u^\xi \right] \\
u^\xi \nabla_\xi u^k &= u^\xi \left[ \pdiff{u^k}{\xi} + \Gamma^k_{\ \xi \alpha} u^\alpha + \Gamma^k_{\ \xi \beta} u^\beta + \Gamma^k_{\ \xi \xi} u^\xi \right]
\end{align}
Pressure gradient:
\begin{align}
\nabla^k p &= g^{k \alpha} \pdiff{p}{\alpha} + g^{k \beta} \pdiff{p}{\beta} + g^{k \xi} \pdiff{p}{\xi}
\end{align}
Coriolis Force:
\begin{align}
f g^{\alpha j} \epsilon_{j \xi k} u^k &= - f g^{\alpha \alpha} J u^\beta + f g^{\alpha \beta} J u^\alpha \\
f g^{\beta j} \epsilon_{j \xi k} u^k &= - f g^{\beta \alpha} J u^\beta + f g^{\beta \beta} J u^\alpha
\end{align}

%\begin{align}
%\vb{u} &= u^\alpha \vb{g}_\alpha + u^\beta \vb{g}_\beta + u^\xi \vb{g}_\xi \\
%&= u^\alpha (\vb{\tilde{g}}_{\alpha} + \frac{1-\xi}{z_{top}-z_s} \pdiff{z_s}{\alpha} \vb{\tilde{g}}_{\xi})
%\end{align}

\subsection{Vertical-orthogonal coordinates}

Vertical-orthogonal coordinates $\alpha, \beta, \xi$ with corresponding terrain-following coordinates $\tilde{\alpha}, \tilde{\beta}, \tilde{\xi}$ with
\begin{align}
\alpha &= \tilde{\alpha}, &\qquad \beta &= \tilde{\beta}, &\qquad \xi &= \xi_s(\alpha, \beta) + (1 - \xi_s(\alpha,\beta)) \tilde{\xi}, \\
\tilde{\alpha} &= \alpha, &\qquad \tilde{\beta} &= \beta, &\qquad \tilde{\xi} &= \frac{\xi - \xi_s(\alpha, \beta)}{1 - \xi_s(\alpha, \beta)}.
\end{align}

\begin{align}
\pdiff{}{\xi} &= \pdiff{\tilde{\xi}}{\xi} \pdiff{}{\tilde{\xi}} = \frac{1}{(1 - \xi_s(\alpha,\beta))} \pdiff{}{\tilde{\xi}}, \\
%\pdiff{\xi_s}{\alpha} &= \pdiff{}{\alpha}\left( \frac{z_s}{z_{top}} \right) = \frac{1}{z_{top}} \pdiff{z_s}{\alpha}, \\
\pdiff{}{\alpha} &= \pdiff{}{\tilde{\alpha}} + \pdiff{\tilde{\xi}}{\alpha} \pdiff{}{\tilde{\xi}} = \pdiff{}{\tilde{\alpha}} - \frac{(1 - \xi)}{(1 - \xi_s(\alpha, \beta))^2} \left( \pdiff{\xi_s}{\alpha} \right) \pdiff{}{\tilde{\xi}}, \\
\pdiff{}{\beta} &= \pdiff{}{\tilde{\beta}} + \pdiff{\tilde{\xi}}{\beta} \pdiff{}{\tilde{\xi}} = \pdiff{}{\tilde{\beta}} - \frac{(1 - \xi)}{(1 - \xi_s(\alpha, \beta))^2} \left( \pdiff{\xi_s}{\beta} \right) \pdiff{}{\tilde{\xi}},
\end{align}

Define:
\begin{align}
A_\alpha &= \pdiff{\tilde{\xi}}{\alpha}, & A_\beta &= \pdiff{\tilde{\xi}}{\beta}, & A_\xi &= \pdiff{\tilde{\xi}}{\xi}
\end{align}

Hydrodynamics (assuming $A_\alpha$, $A_\beta$ and $A_\xi$ are not functions of $\tilde{\xi}$):
\begin{align}
\pdiff{u^\alpha}{t} & + u^\alpha \pdiff{u^\alpha}{\tilde{\alpha}} + u^\beta \pdiff{u^\alpha}{\tilde{\beta}} + \Gamma^{\alpha}_{\ \alpha \alpha} u^\alpha u^\alpha + 2\ \Gamma^{\alpha}_{\ \alpha \beta} u^\alpha u^\beta + \Gamma^{\alpha}_{\ \beta \beta} u^\beta u^\beta \nonumber \\ &  + \frac{1}{\rho} \left[ g^{\alpha \alpha} \pdiff{p}{\tilde{\alpha}} + g^{\alpha \beta} \pdiff{p}{\tilde{\beta}} \right] + f J \left[ - g^{\alpha \alpha} u^\beta + g^{\alpha \beta} u^\alpha \right] \nonumber \\
& + \left[A_\alpha u^\alpha + A_\beta u^\beta + A_\xi u^\xi \right] \pdiff{u^\alpha}{\tilde{\xi}} + \frac{1}{\rho} \left[ g^{\alpha \alpha} A_\alpha + g^{\alpha \beta} A_\beta \right] \pdiff{p}{\tilde{\xi}} = 0, \\
\pdiff{u^\beta}{t} & + u^\alpha \pdiff{u^\beta}{\tilde{\alpha}} + u^\beta \pdiff{u^\beta}{\tilde{\beta}} + \Gamma^{\beta}_{\ \alpha \alpha} u^\alpha u^\alpha + 2\ \Gamma^{\beta}_{\ \alpha \beta} u^\alpha u^\beta + \Gamma^{\beta}_{\ \beta \beta} u^\beta u^\beta \nonumber \\ &  + \frac{1}{\rho} \left[ g^{\beta \alpha} \pdiff{p}{\tilde{\alpha}} + g^{\beta \beta} \pdiff{p}{\tilde{\beta}} \right] + f J \left[ - g^{\beta \alpha} u^\beta + g^{\beta \beta} u^\alpha \right] \nonumber \\
& + \left[A_\alpha u^\alpha + A_\beta u^\beta + A_\xi u^\xi \right] \pdiff{u^\beta}{\tilde{\xi}} + \frac{1}{\rho} \left[ g^{\beta \alpha} A_\alpha + g^{\beta \beta} A_\beta \right] \pdiff{p}{\tilde{\xi}} = 0, \\
\pdiff{\theta}{t} & + u^\alpha \pdiff{\theta}{\tilde{\alpha}} + u^\beta \pdiff{\theta}{\tilde{\beta}} = - \left[ A_\alpha u^\alpha + A_\beta u^\beta + A_\xi u^\xi \right] \pdiff{\theta}{\tilde{\xi}}, \\
\pdiff{u^\xi}{t} & + u^\alpha \pdiff{u^\xi}{\tilde{\alpha}} + u^\beta \pdiff{u^\xi}{\tilde{\beta}} + \left[ A_\alpha u^\alpha + A_\beta u^\beta + A_\xi u^\xi \right] \pdiff{u^\xi}{\tilde{\xi}} = - \frac{1}{\rho} A_\xi \pdiff{p}{\tilde{\xi}}, \\
\pdiff{\rho}{t} & + \frac{1}{J} \pdiff{}{\tilde{\alpha}} \left( J \rho u^\alpha \right) + \frac{1}{J} \pdiff{}{\tilde{\beta}} \left( J \rho u^\beta \right) = - \frac{1}{J} \pdiff{}{\tilde{\xi}} \left[ J \rho \left( A_\alpha u^\alpha + A_\beta u^\beta + A_\xi u^\xi \right) \right].
\end{align}

\section{Horizontal Discretization}

\subsection{Tensor Product Basis}

The tensor-product basis is
\begin{equation} \label{eq:TensorProductBasis}
\phi_{(i,j)}(\alpha, \beta) = \tilde{\phi}_{(i)}(\alpha) \tilde{\phi}_{(j)}(\beta),
\end{equation} where $\tilde{\phi}_{(i)}(x)$ denotes the usual 1D GLL basis function at node $i \in (0, \ldots, n_p)$.  For vector fields, the components of the covariant vector field are given by the tensor-product basis (\ref{eq:TensorProductBasis}).

\subsection{Spectral Element (Scalar Variational Form)}

Consider a scalar conservation law of the form
\begin{equation}
\pdiff{\psi}{t} + \nabla \cdot \vb{F} = S,
\end{equation} with $\vb{F} = F^\alpha \vb{g}_\alpha + F^\beta \vb{g}_\beta$.  Multiplying by the tensor-product basis $\phi_{(p,q)}(\alpha, \beta)$ and integrating over the whole domain yields
\begin{equation} \label{eq:SEMethodDerivation1}
\iint \pdiff{\psi}{t} \phi_{(i,j)} dA = - \iint (\nabla \cdot \vb{F}) \phi_{(i,j)} dA + \iint S \phi_{(i,j)} dA,
\end{equation} where $dA = J d\alpha d\beta$.  Applying integration by parts and using periodicity of the domain then leads to the weak form
\begin{equation}
\iint \pdiff{\psi}{t} \phi_{(i,j)} dA = \iint \nabla \phi_{(i,j)} \cdot \vb{F} dA + \iint S \phi_{(i,j)} dA.
\end{equation}  Further expanding $\psi$ in terms of the horizontal basis functions as
\begin{equation}
\psi(t, \alpha, \beta) = \sum_{(s,t)} \psi_{(s,t)}(t) \phi_{(s,t)}(\alpha, \beta)
\end{equation} leads to
\begin{equation}
\sum_{(s,t)} \pdiff{\psi_{(s,t)}}{t} \iint \phi_{(s,t)} \phi_{(i,j)} dA = \iint \nabla \phi_{(i,j)} \cdot \vb{F} dA + \iint S \phi_{(i,j)} dA.
\end{equation}  Here the vertical dependence of $\psi$ is implicit.  For simplicity we now restrict our domain to a single spectral element (since the DSS procedure will later be used to account for inter-element exchange).  Approximate integration is now applied,
\begin{equation}
\iint f(\alpha, \beta) dA = \Delta \alpha \Delta \beta \sum_{p = 0}^{n_p-1} \sum_{q = 0}^{n_p-1} f(\alpha_p, \beta_q) w_p w_q,
\end{equation} where $w_i$ are the nodal weights of the GLL nodes on the reference element $[0,1]$.  Consequently,
\begin{align}
\label{eq:SEMethodExpansion1} \iint \pdiff{\psi}{t} \phi_{(i,j)} dA &= \Delta \alpha \Delta \beta \pdiff{\psi_{(i,j)}}{t} w_i w_j J(\alpha_i, \beta_j), \\
\label{eq:SEMethodExpansion2} \iint \nabla \phi_{(i,j)} \cdot \vb{F} dA &= \Delta \alpha \Delta \beta w_j \sum_{p = 0}^{n_p-1} \left. \diff{\tilde{\phi}_{(i)}}{\alpha} F^\alpha w_p J \right\vert_{\alpha = \alpha_p, \beta = \beta_j} + \Delta \alpha \Delta \beta w_i \sum_{q = 0}^{n_p-1} \left. \diff{\tilde{\phi}_{(j)}}{\beta} F^\beta w_q J \right\vert_{\alpha = \alpha_i, \beta = \beta_q}, \\
\label{eq:SEMethodExpansion3} \iint S \phi_{(i,j)} dA &= \Delta \alpha \Delta \beta S(\alpha_i, \beta_j) w_i w_j J(\alpha_i, \beta_j).
\end{align}  Substituting (\ref{eq:SEMethodExpansion1})-(\ref{eq:SEMethodExpansion3}) into (\ref{eq:SEMethodDerivation1}) gives the spectral element semi-discretization
\begin{align}
\pdiff{\psi_{(i,j)}}{t} &= \frac{1}{w_i J(\alpha_i, \beta_j)} \sum_{p = 0}^{n_p-1} \left. \diff{\tilde{\phi}_{(i)}}{\alpha} F^\alpha w_p J \right\vert_{\alpha = \alpha_p, \beta = \beta_j} \nonumber \\
& \qquad + \frac{1}{w_j J(\alpha_i, \beta_j)} \sum_{q = 0}^{n_p-1} \left. \diff{\tilde{\phi}_{(j)}}{\beta} F^\beta w_q J \right\vert_{\alpha = \alpha_i, \beta = \beta_q} + S(\alpha_i, \beta_j)
\end{align}

\subsection{Spectral Element (Differential Form)}

The spectral element method can also be derived in differential form by noting that basis functions can be interpreted as components of an interpolating polynomial.  For an arbitrary function $f(\alpha, \beta)$ defined on element nodes we have
\begin{align}
\left. \pdiff{f}{\alpha} \right\vert_{\alpha = \alpha_i, \beta = \beta_j} &= \sum_{p=0}^{n_p-1} \left. \diff{\tilde{\phi}_{(p)}}{\alpha} \right\vert_{\alpha = \alpha_i} f(\alpha_p, \beta_j), \\
\left. \pdiff{f}{\beta} \right\vert_{\alpha = \alpha_i, \beta = \beta_j} &= \sum_{q=0}^{n_p-1} \left. \diff{\tilde{\phi}_{(q)}}{\beta} \right\vert_{\beta = \beta_j} f(\alpha_i, \beta_q).
\end{align}

Second derivatives are defined as follows:
\begin{align}
\left. \frac{\partial^2 f}{\partial \alpha^2} \right\vert_{\alpha = \alpha_i, \beta = \beta_j} &= \sum_{p=0}^{n_p-1} \left. \diff{\tilde{\phi}_{(p)}}{\alpha} \right\vert_{\alpha = \alpha_i} \pdiff{f}{\alpha}(\alpha_p, \beta_j) \\
&= \sum_{p=0}^{n_p-1} \sum_{q=0}^{n_p-1} \left. \diff{\tilde{\phi}_{(p)}}{\alpha} \right\vert_{\alpha = \alpha_i} \left. \diff{\tilde{\phi}_{(q)}}{\alpha} \right\vert_{\alpha = \alpha_p} f(\alpha_q, \beta_j)
\end{align}

\subsection{Flux Reconstruction / Discontinuous Galerkin}

\section{Vertical Discretization}

The following operators will be applied in the context of the vertical discretization:

\begin{center}
\begin{tabular}{cc}
\hline Operator & Description \\
\hline \hline $\mathcal{I}^n_e$ & Interpolate from nodes to edges \\
$\mathcal{I}^e_n$ & Interpolate from edges to nodes \\
$\mathcal{D}^n_e$ & Differentiate from nodes to edges \\
$\mathcal{D}^e_n$ & Differentiate from edges to nodes \\
$\mathcal{D}^n_n$ & Differentiate from nodes to nodes \\
$\mathcal{D}^e_e$ & Differentiate from edges to edges \\
\hline
\end{tabular}
\end{center}

Exner function:
\begin{equation}
\Pi(\rho, \theta) = c_p \left( \frac{R \rho \theta}{p_0} \right)^{R/c_v}
\end{equation}

Vertical discretizations:

\begin{center}
\begin{tabular}{cc|ccc}
\hline Prognostic & & \multicolumn{3}{c}{\underline{Choice of Staggering}} \\
Variable & Operator & DG $(\rho_n \theta_n u^\xi_n)$ & $(\rho_n \theta_n, u^\xi_e)$ & $(\rho_n, u^\xi_e \theta_e)$ \\
\hline \hline & $\displaystyle \Pi_n$ & $\Pi_n(\rho_n, \theta_n)$ & $\Pi_n(\rho_n, \theta_n)$ & $\Pi_n(\rho_n, \mathcal{I}_n^e \theta_e)$ \\[2.5ex]
$\theta$ & $\displaystyle u^\xi \pdiff{\theta}{\xi}$ & $(u^\xi_n) \mathcal{D}_n^n \theta_n$ & $(\mathcal{I}_n^e u^\xi_e) (\mathcal{D}_n^n \theta)$ & $(u^\xi_e) (\mathcal{D}_e^e \theta_e)$ \\[2.5ex]
$u^\xi$ & $\displaystyle \theta \pdiff{\Pi}{\xi}$ & $\theta_n \mathcal{D}^n_n \Pi_n$ & $(\mathcal{I}_n^e \theta_n) (\mathcal{D}^n_e \Pi_n)$ & $\theta_e (\mathcal{D}^n_e \Pi_n)$ \\[2.5ex]
$\rho$ & $\displaystyle \frac{1}{J} \pdiff{}{\xi} (J \rho u^\xi)$ & $\displaystyle \frac{1}{J_n} \mathcal{D}^n_n (J_n \rho_n u^\xi_n)$ & $\displaystyle \frac{1}{J_n} \mathcal{D}^e_n \left[ J_e (\mathcal{I}_e^n \rho_n) u^\xi_e \right]$ & $\displaystyle \frac{1}{J_n} \mathcal{D}^e_n \left[ J_e (\mathcal{I}_e^n \rho_n) u^\xi_e \right]$ \\[2.5ex]
\hline
\end{tabular}
\end{center}

\section{Hyperviscosity}

Hyperviscosity is formulated in variational form.

\subsection{Fourth-Order Scalar Hyperviscosity}

Fourth-order scalar hyperviscosity is implemented using a two stage procedure:
\begin{align}
f &= \mathcal{H}(1) \psi^n, \\
\psi^{n+1} &= \psi^n - \Delta t \mathcal{H}(\nu) f.
\end{align}  The hyperdiffusion operator is defined implicitly via
\begin{align}
f = \mathcal{H}(\nu) \psi \quad \Longleftrightarrow \quad \iint f \phi_{(i,j)} dA = \nu \iint \nabla \phi_{(i,j)} \cdot \nabla \psi dA,
\end{align} where $dA = J d\alpha d\beta$.  Here
\begin{align}
\iint f \phi_{(i,j)} dA = \iint f \tilde{\phi}_{(i)}(\alpha) \tilde{\phi}_{(j)}(\beta) dA = f_{(i,j)} w_i w_j J \Delta \alpha \Delta \beta,
\end{align} and
\begin{align}
\iint \nabla \phi_{(i,j)} \cdot \nabla \psi dA &= \iint g^{pq} \nabla_p \phi \nabla_q \psi dA, \\
&= \iint \pdiff{\phi_{(i,j)}}{\alpha} \left[ g^{\alpha \alpha} \pdiff{\psi}{\alpha} + g^{\alpha \beta} \pdiff{\psi}{\beta} \right] + \pdiff{\phi_{(i,j)}}{\beta} \left[ g^{\beta \alpha} \pdiff{\psi}{\alpha} + g^{\beta \beta} \pdiff{\psi}{\beta} \right] dA, \\
&= \Delta \alpha \Delta \beta \sum_{m=0}^{n_p-1} \sum_{n=0}^{n_p-1} \tilde{\phi}_{(j)} \pdiff{\tilde{\phi}_{(i)}}{\alpha} \left. \left[ g^{\alpha \alpha} \pdiff{\psi}{\alpha} + g^{\alpha \beta} \pdiff{\psi}{\beta} \right] J w_m w_n \right\vert_{\alpha = \alpha_m, \beta = \beta_n} \nonumber \\
& \qquad + \Delta \alpha \Delta \beta \sum_{m=0}^{n_p-1} \sum_{n=0}^{n_p-1} \tilde{\phi}_{(i)} \pdiff{\tilde{\phi}_{(j)}}{\beta} \left. \left[ g^{\beta \alpha} \pdiff{\psi}{\alpha} + g^{\beta \beta} \pdiff{\psi}{\beta} \right] J w_m w_n \right\vert_{\alpha = \alpha_m, \beta = \beta_n} \\
&= \Delta \alpha \Delta \beta w_j \sum_{m=0}^{n_p-1} \pdiff{\tilde{\phi}_{(i)}}{\alpha} \left. \left[ g^{\alpha \alpha} \pdiff{\psi}{\alpha} + g^{\alpha \beta} \pdiff{\psi}{\beta} \right] J w_m \right\vert_{\alpha = \alpha_m, \beta = \beta_j} \nonumber \\
& \qquad + \Delta \alpha \Delta \beta w_i \sum_{n=0}^{n_p-1} \pdiff{\tilde{\phi}_{(j)}}{\beta} \left. \left[ g^{\beta \alpha} \pdiff{\psi}{\alpha} + g^{\beta \beta} \pdiff{\psi}{\beta} \right] J w_n \right\vert_{\alpha = \alpha_i, \beta = \beta_n}
\end{align}  In combination,
\begin{align}
f_{(i,j)} &= \frac{1}{w_i J(\alpha_i, \beta_j)} \sum_{m=0}^{n_p-1} \pdiff{\tilde{\phi}_{(i)}}{\alpha} \left. \left[ g^{\alpha \alpha} \pdiff{\psi}{\alpha} + g^{\alpha \beta} \pdiff{\psi}{\beta} \right] J w_m \right\vert_{\alpha = \alpha_m, \beta = \beta_j} \nonumber \\
& \qquad + \frac{1}{w_j J(\alpha_i, \beta_j)} \sum_{n=0}^{n_p-1} \pdiff{\tilde{\phi}_{(j)}}{\beta} \left. \left[ g^{\beta \alpha} \pdiff{\psi}{\alpha} + g^{\beta \beta} \pdiff{\psi}{\beta} \right] J w_n \right\vert_{\alpha = \alpha_i, \beta = \beta_n}
\end{align}  The derivatives of the scalar field $\psi$ are obtained in the usual manner,
\begin{align}
\left. \pdiff{\psi}{\alpha} \right\vert_{\alpha = \alpha_i, \beta = \beta_j} &= \sum_{m=0}^{n_p-1} \left. \psi_{(m,j)} \pdiff{\tilde{\phi}_{(m)}}{\alpha} \right\vert_{\alpha = \alpha_m, \beta = \beta_j} \\
\left. \pdiff{\psi}{\beta} \right\vert_{\alpha = \alpha_i, \beta = \beta_j} &= \sum_{n=0}^{n_p-1} \left. \psi_{(i,n)} \pdiff{\tilde{\phi}_{(n)}}{\beta} \right\vert_{\alpha = \alpha_i, \beta = \beta_n} \\
\end{align}


\subsection{Fourth-Order Vector Hyperviscosity}

Fourth-order vector hyperviscosity is implemented using a two stage procedure:
\begin{align}
\vb{f} &= \mathcal{H}(1,1) \vb{u}^n, \\
\vb{u}^{n+1} &= \vb{u}^n - \Delta t \mathcal{H}(\nu_d, \nu_v) \vb{f}.
\end{align}  The hyperdiffusion operator is defined implicitly via
\begin{align}
\vb{f} = \mathcal{H}(\nu_d, \nu_v) \vb{u^n} \quad \Longleftrightarrow \quad \iint \vb{f} \cdot \vg{\phi} dA = \iint \nu_d (\nabla \cdot \vg{\phi}) (\nabla \cdot \vb{u}^n) + \nu_v (\nabla \times \vg{\phi})^r (\nabla \times \vb{u}^n)_r dA,
\end{align} where $dA = J d\alpha d\beta$ and
\begin{align}
(\nabla \cdot \vg{\phi}) &= g^{p q} \nabla_p \phi_q = \frac{1}{J} \pdiff{}{\alpha} \left( J g^{\alpha \alpha} \phi_\alpha + J g^{\alpha \beta} \phi_\beta \right) + \frac{1}{J} \pdiff{}{\beta} \left( J g^{\beta \alpha} \phi_\alpha + J g^{\beta \beta} \phi_\beta \right), \\
%(\nabla \cdot \vg{\phi}) &= g^{p q} \nabla_p \phi_q \\
%&= g^{\alpha \alpha} \nabla_\alpha \phi_\alpha + g^{\beta \alpha} \nabla_\beta \phi_\alpha + g^{\alpha \beta} \nabla_\alpha \phi_\beta + g^{\beta \beta} \nabla_\beta \phi_\beta \\
%&= g^{\alpha \alpha} \left[ \pdiff{\phi_\alpha}{\alpha} - \Gamma^{\alpha}_{\ \alpha \alpha} \phi_\alpha - \Gamma^{\beta}_{\ \alpha \alpha} \phi_\beta \right] \\
(\nabla \times \vg{\phi})^r &= \epsilon^{r p q} \nabla_p \phi_q = \epsilon^{r p q} \left[ \pdiff{\phi_q}{x^p} - \Gamma^{k}_{\ p q} \phi_k \right] = \frac{1}{J} \left[ \pdiff{\phi_\beta}{\alpha} - \pdiff{\phi_\alpha}{\beta} \right].
\end{align}

Here we assume that $(\nabla \cdot \vb{u})$ and $(\nabla \times \vb{u})_r$ (covariant radial component of curl) have already been computed.

\subsubsection{Vector basis with zero $\beta$ component}

If $\phi_{(i,j) \alpha} = \tilde{\phi}_{(i)}(\alpha) \tilde{\phi}_{(j)}(\beta)$ and $\phi_{(i,j) \beta} = 0$ then
\begin{align} \label{eq:VecHyperviscosityZeroBetaLHS}
\iint \vb{f} \cdot \vb{\phi} dA = \iint f^\alpha \tilde{\phi}_{(i)}(\alpha) \tilde{\phi}_{(j)}(\beta) dA = f^\alpha_{(i,j)} w_i w_j J \Delta \alpha \Delta \beta
\end{align}

The divergent term is defined by
\begin{align}
(\nabla \cdot \vg{\phi}_{(i,j)}) &= \frac{1}{J} \pdiff{}{\alpha} \left( J g^{\alpha \alpha} \phi_{(i,j)\alpha} \right) + \frac{1}{J} \pdiff{}{\beta} \left( J g^{\beta \alpha} \phi_{(i,j)\alpha} \right) \\
%&= \phi_{(i,j)\alpha} \left[ \frac{1}{J} \pdiff{}{\alpha}\left( J g^{\alpha \alpha} \right) + \frac{1}{J} \pdiff{}{\beta}\left( J g^{\beta \alpha} \right) \right] + g^{\alpha \alpha} \pdiff{\phi_{(i,j)\alpha}}{\alpha} + g^{\beta \alpha} \pdiff{\phi_{(i,j)\beta}}{\beta}
&= \frac{\tilde{\phi}_{(j)}(\beta)}{J} \pdiff{}{\alpha} \left( J g^{\alpha \alpha} \tilde{\phi}_{(i)}(\alpha) \right)  + \frac{\tilde{\phi}_{(i)}(\alpha)}{J} \pdiff{}{\beta} \left( J g^{\beta \alpha} \tilde{\phi}_{(j)}(\beta) \right),
\end{align} and
\begin{align}
& \iint (\nabla \cdot \vg{\phi}_{(i,j)}) (\nabla \cdot \vb{u}) dA \nonumber \\
%& \qquad = \Delta \alpha \Delta \beta \sum_{m=0}^{n_p-1} \sum_{n=0}^{n_p-1} \left\{ \tilde{\phi}_{(i)}(\alpha_m) \tilde{\phi}_{(j)}(\beta_n) \left[ \frac{1}{J} \pdiff{}{\alpha}\left( J g^{\alpha \alpha} \right) + \frac{1}{J} \pdiff{}{\beta}\left( J g^{\beta \alpha} \right) \right] \right. \nonumber \\
%& \qquad \qquad \left. + g^{\alpha \alpha} \tilde{\phi}_{(j)}(\beta_n) \diff{\tilde{\phi}_{(i)}}{\alpha} + \tilde{\phi}_{(i)}(\alpha_m) g^{\beta \alpha} \diff{\tilde{\phi}_{(j)}}{\beta} \right\} (\nabla \cdot \vb{u}) J w_m w_n \\
%& \qquad = \Delta \alpha \Delta \beta w_i w_j \left. \left[ \pdiff{}{\alpha}\left( J g^{\alpha \alpha} \right) + \pdiff{}{\beta}\left( J g^{\beta \alpha} \right) \right] (\nabla \cdot \vb{u}) \right\vert_{\alpha = \alpha_i, \beta = \beta_j} \nonumber \\
%& \qquad \qquad + \Delta \alpha \Delta \beta w_j \sum_{m = 0}^{n_p-1} \left. J g^{\alpha \alpha} \diff{\tilde{\phi}_{(i)}}{\alpha} (\nabla \cdot \vb{u}) w_m \right\vert_{\alpha = \alpha_m, \beta = \beta_j} \nonumber \\
%& \qquad \qquad + \Delta \alpha \Delta \beta w_i \sum_{n = 0}^{n_p-1} \left. J g^{\beta \alpha} \diff{\tilde{\phi}_{(j)}}{\beta} (\nabla \cdot \vb{u}) w_n \right\vert_{\alpha = \alpha_i, \beta = \beta_n}
& \qquad = \Delta \alpha \Delta \beta \sum_{m=0}^{n_p-1} \sum_{n=0}^{n_p-1} \left[ \frac{\tilde{\phi}_{(j)}(\beta_n)}{J} \pdiff{}{\alpha} \left( J g^{\alpha \alpha} \tilde{\phi}_{(i)}(\alpha) \right)  + \frac{\tilde{\phi}_{(i)}(\alpha_m)}{J} \pdiff{}{\beta} \left( J g^{\beta \alpha} \tilde{\phi}_{(j)}(\beta) \right) \right] (\nabla \cdot \vb{u}) J w_m w_n \\
& \qquad = \Delta \alpha \Delta \beta w_j \sum_{m = 0}^{n_p-1} \left. \pdiff{}{\alpha} \left( J g^{\alpha \alpha} \tilde{\phi}_{(i)}(\alpha) \right) (\nabla \cdot \vb{u}) w_m \right\vert_{\alpha = \alpha_m, \beta = \beta_j} \nonumber \\
& \qquad \qquad + \Delta \alpha \Delta \beta w_i \sum_{n = 0}^{n_p-1} \left. \pdiff{}{\beta} \left( J g^{\beta \alpha} \tilde{\phi}_{(j)}(\beta) \right) (\nabla \cdot \vb{u}) w_n \right\vert_{\alpha = \alpha_i, \beta = \beta_n} \\
& \qquad = \Delta \alpha \Delta \beta w_j \sum_{m = 0}^{n_p-1} \left. J g^{\alpha \alpha} \diff{\tilde{\phi}_{(i)}}{\alpha} (\nabla \cdot \vb{u}) w_m \right\vert_{\alpha = \alpha_m, \beta = \beta_j} \nonumber \\
& \qquad \qquad + \Delta \alpha \Delta \beta w_i \sum_{n = 0}^{n_p-1} \left. J g^{\beta \alpha} \diff{\tilde{\phi}_{(j)}}{\beta} (\nabla \cdot \vb{u}) w_n \right\vert_{\alpha = \alpha_i, \beta = \beta_n} \label{eq:VecHyperviscosityZeroBetaDiv}
\end{align}

Further, the vortical term is defined by
\begin{align}
(\nabla \times \vg{\phi}_{(i,j)})^r &= - \frac{1}{J} \pdiff{\phi_{(i,j) \alpha}}{\beta} = - \frac{\tilde{\phi}_{(i)}}{J} \diff{\tilde{\phi}_{(j)}}{\beta}
\end{align} and so
\begin{align}
& \iint (\nabla \times \vg{\phi}_{(i,j)})^r (\nabla \times \vb{u})_r dA \nonumber \\
& \qquad = \Delta \alpha \Delta \beta \sum_{m=0}^{n_p-1} \sum_{n=0}^{n_p-1} \left. \left[- \frac{\tilde{\phi}_{(i)}(\alpha_m)}{J} \diff{\tilde{\phi}_{(j)}}{\beta} \right] (\nabla \times \vb{u})_r J w_m w_n \right\vert_{\alpha = \alpha_m, \beta = \beta_n} \\
& \qquad = - \Delta \alpha \Delta \beta w_i \sum_{n=0}^{n_p-1} \left. \diff{\tilde{\phi}_{(j)}}{\beta} (\nabla \times \vb{u})_r w_n \right\vert_{\alpha = \alpha_i, \beta = \beta_n} \label{eq:VecHyperviscosityZeroBetaCurl}
\end{align}

Combining (\ref{eq:VecHyperviscosityZeroBetaLHS}), (\ref{eq:VecHyperviscosityZeroBetaDiv}) and (\ref{eq:VecHyperviscosityZeroBetaCurl}) then gives
\begin{align}
f^\alpha_{(i,j)} &= \frac{\nu_d}{J(\alpha_i, \beta_j) w_i} \sum_{m = 0}^{n_p-1} \left. J g^{\alpha \alpha} \diff{\tilde{\phi}_{(i)}}{\alpha} (\nabla \cdot \vb{u}) w_m \right\vert_{\alpha = \alpha_m, \beta = \beta_j} \nonumber \\
& \qquad + \frac{\nu_d}{J(\alpha_i, \beta_j) w_j} \sum_{n = 0}^{n_p-1} \left. J g^{\beta \alpha} \diff{\tilde{\phi}_{(j)}}{\beta} (\nabla \cdot \vb{u}) w_n \right\vert_{\alpha = \alpha_i, \beta = \beta_n} \nonumber \\
& \qquad - \frac{\nu_v}{J(\alpha_i, \beta_j) w_j} \sum_{n=0}^{n_p-1} \left. \diff{\tilde{\phi}_{(j)}}{\beta} (\nabla \times \vb{u})_r w_n \right\vert_{\alpha = \alpha_i, \beta = \beta_n}
\end{align}

\subsubsection{Vector basis with zero $\alpha$ component}

If $\phi_{(i,j) \alpha} = 0$ and $\phi_{(i,j) \beta} = \tilde{\phi}_{(j)}(\alpha) \tilde{\phi}_{(j)}(\beta)$ then
\begin{align} \label{eq:VecHyperviscosityZeroAlphaLHS}
\iint \vb{f} \cdot \vb{\phi} dA = \iint f^\beta \tilde{\phi}_{(i)}(\alpha) \tilde{\phi}_{(j)}(\beta) dA = f^\beta_{(i,j)} w_i w_j J \Delta \alpha \Delta \beta
\end{align}

The divergent term is defined by
\begin{align}
(\nabla \cdot \vg{\phi}_{(i,j)}) &= \frac{1}{J} \pdiff{}{\alpha} \left( J g^{\alpha \beta} \phi_{(i,j) \beta} \right) + \frac{1}{J} \pdiff{}{\beta} \left( J g^{\beta \beta} \phi_{(i,j) \beta} \right) \\
%&= \phi_{(i,j) \beta} \left[ \frac{1}{J} \pdiff{}{\alpha} \left( J g^{\alpha \beta} \right) + \frac{1}{J} \pdiff{}{\beta} \left( J g^{\beta \beta} \right) \right] + g^{\alpha \beta} \pdiff{\phi_{(i,j) \beta}}{\alpha} + g^{\beta \beta} \pdiff{\phi_{(i,j) \beta}}{\beta}
&= \frac{\tilde{\phi}_{(j)}(\beta)}{J} \pdiff{}{\alpha} \left( J g^{\alpha \beta} \tilde{\phi}_{(i)} \right) + \frac{\tilde{\phi}_{(i)}(\alpha)}{J} \pdiff{}{\beta} \left( J g^{\beta \beta} \tilde{\phi}_{(j)} \right),
\end{align} and
\begin{align}
& \iint (\nabla \cdot \vg{\phi}_{(i,j)}) (\nabla \cdot \vb{u}) dA \nonumber \\
%& \qquad = \Delta \alpha \Delta \beta \sum_{m=0}^{n_p-1} \sum_{n=0}^{n_p-1} \left\{ \tilde{\phi}_{(i)} \tilde{\phi}_{(j)} \left[ \frac{1}{J} \pdiff{}{\alpha} \left( J g^{\alpha \beta} \right) + \frac{1}{J} \pdiff{}{\beta} \left( J g^{\beta \beta} \right) \right] \right. \nonumber \\
%& \qquad \qquad \left. + \tilde{\phi}_{(j)} g^{\alpha \beta} \diff{\tilde{\phi}_{(i)}}{\alpha} + \tilde{\phi}_{(i)} g^{\beta \beta} \diff{\tilde{\phi}_{(j)}}{\beta} \right\} (\nabla \cdot \vb{u}) J w_m w_n \\
%& \qquad = \Delta \alpha \Delta \beta w_i w_j \left. \left[ \frac{1}{J} \pdiff{}{\alpha} \left( J g^{\alpha \beta} \right) + \frac{1}{J} \pdiff{}{\beta} \left( J g^{\beta \beta} \right) \right] (\nabla \cdot \vb{u}) \right\vert_{\alpha = \alpha_i, \beta = \beta_j} \nonumber \\
%& \qquad \qquad + \Delta \alpha \Delta \beta w_j \sum_{m = 0}^{n_p-1} \left. J g^{\alpha \beta} \diff{\tilde{\phi}_{(i)}}{\alpha} (\nabla \cdot \vb{u}) w_m \right\vert_{\alpha = \alpha_m, \beta = \beta_j} \nonumber \\
%& \qquad \qquad + \Delta \alpha \Delta \beta w_i \sum_{n = 0}^{n_p-1} \left. J g^{\beta \beta} \diff{\tilde{\phi}_{(j)}}{\beta} (\nabla \cdot \vb{u}) w_n \right\vert_{\alpha = \alpha_i, \beta = \beta_n}
& \qquad = \Delta \alpha \Delta \beta \sum_{m=0}^{n_p-1} \sum_{n=0}^{n_p-1} \left[ \frac{\tilde{\phi}_{(j)}(\beta_n)}{J} \pdiff{}{\alpha} \left( J g^{\alpha \beta} \tilde{\phi}_{(i)} \right) + \frac{\tilde{\phi}_{(i)}(\alpha_m)}{J} \pdiff{}{\beta} \left( J g^{\beta \beta} \tilde{\phi}_{(j)} \right) \right] (\nabla \cdot \vb{u}) J w_m w_n \\
& \qquad = \Delta \alpha \Delta \beta w_j \sum_{m=0}^{n_p-1} \left. \pdiff{}{\alpha} \left( J g^{\alpha \beta} \tilde{\phi}_{(i)}(\alpha) \right) (\nabla \cdot \vb{u}) w_m \right\vert_{\alpha = \alpha_m, \beta = \beta_j} \nonumber \\
& \qquad \qquad + \Delta \alpha \Delta \beta w_i \sum_{n=0}^{n_p-1} \left. \pdiff{}{\beta} \left( J g^{\beta \beta} \tilde{\phi}_{(j)}(\beta) \right) (\nabla \cdot \vb{u}) w_n \right\vert_{\alpha = \alpha_i, \beta = \beta_n} \\
& \qquad = \Delta \alpha \Delta \beta w_j \sum_{m=0}^{n_p-1} \left. J g^{\alpha \beta} \diff{\tilde{\phi}_{(i)}}{\alpha} (\nabla \cdot \vb{u}) w_m \right\vert_{\alpha = \alpha_m, \beta = \beta_j} \nonumber \\
& \qquad \qquad + \Delta \alpha \Delta \beta w_i \sum_{n=0}^{n_p-1} \left. J g^{\beta \beta} \diff{\tilde{\phi}_{(j)}}{\beta} (\nabla \cdot \vb{u}) w_n \right\vert_{\alpha = \alpha_i, \beta = \beta_n} \label{eq:VecHyperviscosityZeroAlphaDiv}
\end{align}

Further, the vortical term is defined by
\begin{align}
(\nabla \times \vg{\phi}_{(i,j)})^r &= \frac{1}{J} \pdiff{\phi_{(i,j) \beta}}{\alpha} = \frac{\tilde{\phi}_{(j)}}{J} \diff{\tilde{\phi}_{(i)}}{\alpha}
\end{align} and so
\begin{align}
& \iint (\nabla \times \vg{\phi}_{(i,j)})^r (\nabla \times \vb{u})_r dA \nonumber \\
& \qquad = \Delta \alpha \Delta \beta \sum_{m=0}^{n_p-1} \sum_{n=0}^{n_p-1} \left. \left[\frac{\tilde{\phi}_{(j)}(\beta_n)}{J} \diff{\tilde{\phi}_{(i)}}{\alpha}\right] (\nabla \times \vb{u})_r J w_m w_n \right\vert_{\alpha = \alpha_m, \beta = \beta_n} \\
& \qquad = \Delta \alpha \Delta \beta w_j \sum_{m=0}^{n_p-1} \left. \diff{\tilde{\phi}_{(i)}}{\alpha} (\nabla \times \vb{u})_r w_m \right\vert_{\alpha = \alpha_m, \beta = \beta_j} \label{eq:VecHyperviscosityZeroAlphaCurl}
\end{align}

Combining (\ref{eq:VecHyperviscosityZeroAlphaLHS}), (\ref{eq:VecHyperviscosityZeroAlphaDiv}) and (\ref{eq:VecHyperviscosityZeroAlphaCurl}) then gives
\begin{align}
f^{\beta}_{(i,j)} &= \frac{\nu_d}{J(\alpha_i, \beta_j) w_i} \sum_{m=0}^{n_p-1} \left. J g^{\alpha \beta} \diff{\tilde{\phi}_{(i)}}{\alpha} (\nabla \cdot \vb{u}) w_m \right\vert_{\alpha = \alpha_m, \beta = \beta_j} \nonumber \\
& \qquad + \frac{\nu_d}{J(\alpha_i, \beta_j) w_j} \sum_{n=0}^{n_p-1} \left. J g^{\beta \beta} \diff{\tilde{\phi}_{(j)}}{\beta} (\nabla \cdot \vb{u}) w_n \right\vert_{\alpha = \alpha_i, \beta = \beta_n} \nonumber \\
& \qquad + \frac{\nu_v}{J(\alpha_i, \beta_j) w_i} \sum_{m=0}^{n_p-1} \left. \diff{\tilde{\phi}_{(i)}}{\alpha} (\nabla \times \vb{u})_r w_m \right\vert_{\alpha = \alpha_m, \beta = \beta_j}
\end{align}

\end{document}